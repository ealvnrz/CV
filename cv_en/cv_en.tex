\documentclass[11pt,a4paper,roman]{moderncv}        % possible options include font size ('10pt', '11pt' and '12pt'), paper size ('a4paper', 'letterpaper', 'a5paper', 'legalpaper', 'executivepaper' and 'landscape') and font family ('sans' and 'roman')

% moderncv themes
\moderncvstyle{classic}                             % style options are 'casual' (default), 'classic', 'banking', 'oldstyle' and 'fancy'
\moderncvcolor{grey}                               % color options 'black', 'blue' (default), 'burgundy', 'green', 'grey', 'orange', 'purple' and 'red'

% character encoding
\usepackage[utf8]{inputenc}                       % if you are not using xelatex ou lualatex, replace by the encoding you are using
\usepackage[T1]{fontenc}
%\usepackage{lmodern}
\usepackage[english]{babel}  % FIXME: using spanish breaks moderncv
\usepackage{bold-extra}
\usepackage{xcolor}
\usepackage{lastpage}
\usepackage{fontawesome5}

\renewcommand{\familydefault}{\rmdefault}         % to set the default font; use '\sfdefault' for the default sans serif font, '\rmdefault' for the default roman one, or any tex font name
\definecolor{color1}{rgb}{0.52, 0.52, 0.51}
\setlength{\hintscolumnwidth}{2cm}

\renewcommand*{\sectionfont}{\bfseries\scshape}
\renewcommand*{\sectionstyle}[1]{{\sectionfont\textcolor{black}{#1}}}


% adjust the page margins
\usepackage[a4paper, left=1cm, top=2.5cm, bottom=2.5cm, right=1cm]{geometry}
%\setlength{\hintscolumnwidth}{3cm}                % if you want to change the width of the column with the dates
%\setlength{\makecvheadnamewidth}{10cm}            % for the 'classic' style, if you want to force the width allocated to your name and avoid line breaks. be careful though, the length is normally calculated to avoid any overlap with your personal info; use this at your own typographical risks...

\newcommand{\gscholarIcon}{\raisebox{-0.1cm}{\includegraphics[height=\baselineskip]{google-scholar.png}}}
% personal data
\name{}{}
\title{}                               % optional, remove / comment the line if not wanted
%\address{street and number}{postcode city}{country}% optional, remove / comment the line if not wanted; the "postcode city" and "country" arguments can be omitted or provided empty
%\phone[mobile]{+1~(234)~567~890}                   % optional, remove / comment the line if not wanted; the optional "type" of the phone can be "mobile" (default), "fixed" or "fax"
%\phone[fixed]{+2~(345)~678~901}
%\phone[fax]{+3~(456)~789~012}
%\email{john@doe.org}                               % optional, remove / comment the line if not wanted
%\homepage{www.johndoe.com}                         % optional, remove / comment the line if not wanted
%\social[linkedin]{john.doe}                        % optional, remove / comment the line if not wanted
%\social[xing]{john\_doe}                           % optional, remove / comment the line if not wanted
%\social[twitter]{jdoe}                             % optional, remove / comment the line if not wanted
%\social[github]{jdoe}                              % optional, remove / comment the line if not wanted
%\social[gitlab]{jdoe}                              % optional, remove / comment the line if not wanted
%\social[skype]{jdoe}                               % optional, remove / comment the line if not wanted
%\extrainfo{additional information}                 % optional, remove / comment the line if not wanted
%\photo[64pt][0.4pt]{picture}                       % optional, remove / comment the line if not wanted; '64pt' is the height the picture must be resized to, 0.4pt is the thickness of the frame around it (put it to 0pt for no frame) and 'picture' is the name of the picture file
%\quote{Some quote}                                 % optional, remove / comment the line if not wanted

% bibliography adjustements (only useful if you make citations in your resume, or print a list of publications using BibTeX)
%   to show numerical labels in the bibliography (default is to show no labels)
%\makeatletter\renewcommand*{\bibliographyitemlabel}{\@biblabel{\arabic{enumiv}}}\makeatother
\renewcommand*{\bibliographyitemlabel}{\arabic{enumiv}.}
\renewcommand*\descriptionlabel[1]{\hspace\labelsep\normalfont #1}
%   to redefine the bibliography heading string ("Publications")
%\renewcommand{\refname}{Articles}

% bibliography with mutiple entries
%\usepackage{multibib}
%\newcites{book,misc}{{Books},{Others}}
%----------------------------------------------------------------------------------
%            content
%----------------------------------------------------------------------------------
\AtEndPreamble{
  \hypersetup{
    colorlinks=false,
    frenchlinks=false,
    urlcolor=red,
    urlbordercolor=cyan,
    pdfborder = {1 1 1}
}}

\nopagenumbers{}
\renewcommand{\headrulewidth}{0.4pt}
\usepackage{fancyhdr}
\pagestyle{fancy}

\fancyhead[L]{Eloy Alvarado Narváez}
\fancyhead[R]{\thepage\  of \pageref{LastPage}}
\begin{document}
\thispagestyle{empty}

{\bfseries \LARGE Eloy Alvarado Narváez}\\
\href{http://industrias.usm.cl}{Departmento de Industrias\\ Universidad Técnica Federico Santa María} \\ Santiago, Chile.\\
Office: A-293\\
\href{mailto:eloy.alvarado@usm.cl}{\faIcon[regular]{envelope}}
\href{https://github.com/ealvnrz}{\faIcon{github}}
\href{https://www.linkedin.com/in/ealvnrz/}{\faIcon{linkedin}}
\href{https://scholar.google.cl/citations?user=iO2zYZoAAAAJ&hl=es}{\gscholarIcon} 
\href{https://orcid.org/0000-0001-7522-2327}{\faOrcid}

%\makecvtitle
\section{Employment}

\cvitemwithcomment{2023 -- \hfill}{Academic Instructor, \href{http://www.usm.cl}{Universidad Técnica Federico Santa María}}{Santiago, Chile.}
\cvitemwithcomment{2022 -- 2022}{Lecturer, \href{http://www.usm.cl}{Universidad Técnica Federico Santa María}}{Santiago, Chile.}
\cvitemwithcomment{2021 -- 2022}{External Adviser, \href{http://www.ifop.cl}{Instituto de Fomento Pesquero}}{Valparaíso, Chile.}
\cvitemwithcomment{2019 -- 2022}{Lecturer, \href{https://www.uv.cl/}{Universidad de Valparaíso}}{Valparaíso, Chile.}
\cvitemwithcomment{2017 -- 2017}{Lecturer, \href{http://www.pucv.cl}{Pontificia Universidad Católica de Valparaíso}}{Valparaíso,Chile.}
\cvitemwithcomment{2016 -- 2017}{Statistical Analyst, \href{http://www.equifax.com}{Equifax Inc.}}{Santiago, Chile. / Atlanta, US.}

\section{Education}
\cvitemwithcomment{2018 -- 2022}{Ph.D. in Statistics, \href{https://www.uv.cl/}{Instituto de Estadística, Universidad de Valparaíso } }{Valparaíso, Chile.}
\cvitem{}{Advisors: Moreno Bevilacqua and Christian Caamaño}

\cvitemwithcomment{2013 -- 2015}{Bachelor in Statistics / Statistician, \href{http://www.pucv.cl}{Pontificia Universidad Católica de Valparaíso}}{Valparaíso, Chile.}
\cvitemwithcomment{2014}{Exchange program in \href{http://www.korea.ac.kr}{Korea University}}{Seoul, South Korea}

\section{Languages}
\cvitem{Spanish}{Native Speaker}
\cvitem{English}{Full professional proficiency, \href{https://www.ets.org/toefl.html}{TOEFL iBT} 90 (November, 2015)}
\cvitem{French}{Beginner A1, \href{https://www.institutofrances.cl}{Institut Français}}
\cvitem{Korean}{Beginner A2, \href{https://www.korea.edu/}{Korea University}}



% Publications from a BibTeX file without multibib
%  for numerical labels: \renewcommand{\bibliographyitemlabel}{\@biblabel{\arabic{enumiv}}}% CONSIDER MERGING WITH PREAMBLE PART
%  to redefine the heading string ("Publications"): \renewcommand{\refname}{Articles}
\renewcommand{\refname}{Research Articles}
\bibliographystyle{unsrt}
\nocite{*}
\bibliography{publications}                        % 'publications' is the name of a BibTeX file

%\section{Under peer review articles}

%\cvitem{1.}{Moreno Bevilacqua, Eloy Alvarado, Christian Caamaño. \textit{``A flexible clayton-like spatial copula with application to bounded support data''}. Journal of Multivariate Analysis.}

\section{Grants and Projects}

\cvitem{2023 -- 2024}{Modeling and estimating massive point referenced spatial data, MATH-AMSUD AMSUD220041. Research Associate.}
\cvitem{2018 -- 2022}{National Doctoral Degree, ANID. No. 2018-21180953}
\cvitem{2020 -- 2020}{Semi-presential technological support platform for urgent and priority dental care for the elderly in the context
of the COVID-19 pandemic in the Chilean population, ANID, Reference No. COVID0766, Chile. Collaborator}

\section{Teaching}

% USM
\cvitem{}{\href{https://www.usm.cl/}{\textbf{\textsc{Universidad Técnica Federico Santa María}}}}
\cvitem{2024/01}{PII 402: Stochastic Processes (Postgraduate)}
\cvitem{2023/02}{PII 402: Stochastic Processes (Postgraduate)}
\cvitem{2023/01}{ICN 292:  Management Information Systems}
\cvitem{2023/01}{IND 163C: Business Analytics}
\cvitem{2022/02}{IND 163C: Business Analytics}

% UNAB
\cvitem{}{\href{https://www.unab.cl/}{\textbf{\textsc{Universidad Andrés Bello}}}}
\cvitem{2024/01}{CIND 217: Inteligencia de Negocios}
\cvitem{2023/02}{IICG P01: Inteligencia de Negocios}

%UV
\cvitem{}{\href{https://www.uv.cl/}{\textbf{\textsc{Universidad de Valparaíso}}}}
\cvitem{2022/02}{IEST 422: Statistical Quality Control}
\cvitem{2022/02}{LFIS 325: Statistics for Physical Sciences}
\cvitem{2022/01}{IECD 411: Machine Learning}
\cvitem{2022/01}{IEST 414: Time Series}
\cvitem{2022/01}{IECD 415: Multivariate Methods}
\cvitem{2021/02}{IECD 325: Linear Models and Experimental Design}
\cvitem{2021/02}{LFIS 325: Statistics for Physical Sciences}
\cvitem{2021/02}{IEST 412: Sampling Theory II}
\cvitem{2021/01}{IEST 322: Sampling Theory I}
\cvitem{2021/01}{IEST 412: Sampling Theory II}
\cvitem{2021/01}{IEC 311: Probability and Statistics}
\cvitem{2020/02}{IEST 322: Sampling Theory I}
\cvitem{2020/02}{LFIS 325: Statistics for Physical Sciences}
\cvitem{2020/01}{IEST 412: Sampling Theory II}
\cvitem{2020/01}{IEC 311: Probability and Statistics}
\cvitem{2019/02}{IEST 322: Sampling Theory I}
\cvitem{2019/02}{IEST 213: Statistical Methods}
\cvitem{}{\href{http://www.pucv.cl}{\textbf{\textsc{Pontificia Universidad Católica de Valparaíso}}}}
\cvitem{2017/02}{EST 205: Probability and Statistics}
\cvitem{2017/01}{EST 205: Probability and Statistics}


\section{Advising}
\cvitem{2023}{Ricardo Menares. Statistics and Data Science Engineering. \href{https://www.uv.cl/}{Universidad de Valparaíso}. Co-advisor.}

\cvitem{2022}{Cristóbal Collao. Statistical Engineering. \href{https://www.uv.cl/}{Universidad de Valparaíso}. Co-advisor.}


\section{Scientific Events}

\cvitem{2023}{\textit{``Herramientas y estrategias para una investigación reproducible''}. 3rd Open Science Webinar USM. Organizer: \href{https://www.usm.cl/}{Universidad Técnica Federico Santa María}}


\cvitem{2023}{\textit{``Retroalimentación basada en IA para estudiantes de Ingeniería''}. XXXV Chilean Congress of Engineering Education. Organizer: \href{https://www.sochedi.cl}{Chilean Society of Engineering Education}}
\cvitem{2022}{\textit{``A flexible clayton-like spatial copula with application to bounded support data''}. III Statistics Workshop: Graduate Contributions and Scientific initiation. Organizer: \href{https://www.soche.cl}{Chilean Statistical Society}}
\cvitem{2015}{\textit{``Stochastics differential equations in mathematical finance: The Black-Scholes Model''}. XXI Valparaíso Statistical Week. \href{https://www.pucv.cl}{Pontificia Universidad Católica de Valparaíso}}


\section{References}
\cvlistitem{Moreno Bevilacqua, Associate Professor at \href{https:/www.uai.cl}{Universidad Adolfo Ibañez}.  \href{mailto:moreno.bevilacqua@uai.cl}{\texttt{moreno.bevilacqua@uai.cl}}}
\cvlistitem{Christian Caamaño, Associate professor at \href{https://www.ubb.cl}{Universidad del Bío-Bío}. \href{mailto:chcaaman@ubiobio.cl}{\texttt{chcaaman@ubiobio.cl}}}
\cvlistitem{Carlos Montenegro, Head of Fisheries Research at \href{https://www.ifop.cl}{IFOP}.
\href{mailto:carlos.montenegro@ifop.cl}{\texttt{carlos.montenegro@ifop.cl}}}

\vfill
\textit{Updated: March 2024}
% Publications from a BibTeX file using the multibib package
%\section{Publications}
%\nocitebook{book1,book2}
%\bibliographystylebook{plain}
%\bibliographybook{publications}                   % 'publications' is the name of a BibTeX file
%\nocitemisc{misc1,misc2,misc3}
%\bibliographystylemisc{plain}
%\bibliographymisc{publications}                   % 'publications' is the name of a BibTeX file
\end{document}


%% end of file `template.tex'.